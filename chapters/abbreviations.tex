%%%Formelzeichenverzeichnis
\addsec*{Formelzeichen und Abkürzungen}
\markright{Formelzeichen und Abkürzungen}
\pagestyle{scrheadings}
%\begin{tabular}{l@{\hspace{20mm}} l@{\hspace{20mm}} l}
\begin{tabular}{p{40mm} p{30mm} p{73mm}}
\bfseries Formelzeichen & \bfseries Einheit & \bfseries Bezeichnung\\
$A$&mm\textsuperscript{2}&Querschnittsfläche\\
$A_{D}$&mm\textsuperscript{2}&Drosselquerschnittsfläche\\
$A_{i}$&kN&Amplitude\\
$A_{K}$&mm\textsuperscript{2}&Kolbenquerschnittsfläche\\
$A_{min}$&mm\textsuperscript{2}&minimale Querschnittsfläche\\
$A\textsubscript{eff}$&mm\textsuperscript{2}&effektive Fläche\\
$A_{eff}$&mm\textsuperscript{2}&effektive Fläche\\
$b$&mm&Abströmbreite\\
$c_F$&mm N\textsuperscript{-1}&Federsteifigkeit\\
$d$&mm&Kolbendurchmesser Steuergeometrie\\
$d_K$&mm&Kolbennenndurchmesser\\
$E_{Oel}$&bar&Kompressionsmodul Hydrauliköl\\
$f$&Hz&Frequenz\\
$F$&N&Kraft\\
$F_{F}$&N&Federkraft\\
$F_{G}$&N&Gesamtkraft\\
$h$&µm&Lagerspalthöhe\\
$\Delta{h}$&µm&Änderung Lagerspalthöhe\\
$K$&-&Korrekturfaktor\\
$l$&mm&Abströmlänge\\
$l_K$&mm&Länge der Steuergeometrie\\
$m$&kg&Masse\\
$p_{P}$&bar&Pumpendruck\\
$p_{T}$&bar&Taschendruck\\
$p_{U}$&bar&Umgebungsdruck\\
$\Delta{p}$&bar&Druckdifferenz\\
$\Delta{p1}$&bar&Druckdifferenz zwischen Pumpendruck und Umgebungsdruck\\
$\Delta{p2}$&bar&Druckdifferenz zwischen Pumpendruck und Taschendruck\\
$Q$&m\textsuperscript{3}\,s\textsuperscript{-1}&Volumenstrom\\
$Q_T$&m\textsuperscript{3}\,s\textsuperscript{-1}&Taschenvolumenstrom\\
$Ra$&µm&Mittenrauwert\\
$r$&mm&Kolbenradius Steuergeometrie\\
$T_{Oel}$&°C&Öltemperatur\\
$t$&s&Zeit\\
$V$&m³&Volumen\\
$V_0$&m³&Ausgangsvolumen\\
$\Delta{V}$&m³&Volumenänderung\\
$W$&Js\textsuperscript{-1}&Elektrische Leistung\\
\end{tabular}
\newpage
\begin{tabular}{p{40mm} p{30mm} p{73mm}}
%\bfseries Formelzeichen & \bfseries Einheit & \bfseries Bezeichnung\\
$x$&mm&Abstand Kolbenreferenzpunkt zu Steuerkante\\
$\eta$&mPas&dynamische Viskosität\\
$\varepsilon$&m&Sandkörnungsrauhigkeit\\
$\nu$&mm\textsuperscript{2}\,s\textsuperscript{-1}&kinematische Viskosität\\
\vspace{1 cm}
\end{tabular}
%\clearpage
%\addsec*{Abkürzungen}
%\markright{Abkürzungen}

\pagestyle{scrheadings}
\begin{tabular}{p{40mm} p{110mm}}
\bfseries Abkürzung & \bfseries Bezeichnung\\
CAD&Computer Aided Design\\
CNC&Computerized Numerical Control\\
CPU&Central Prozessing Unit\\
DBV&Druckbegrenzungsventil\\
DMV&Druckminderventil\\
GHz&Gigahertz\\
ISO VG 46, HLP46&Mineralisches Standard Hydrauliköl mit Hochdruckzusatz und einer kinematischen Viskosität von 46 mm\textsuperscript{2}\,s\textsuperscript{-1} bei 40\,°C\\
Mio.&Million\\
MP&Messpunkt\\
PM&Progressivmengenregler\\
usw&und so weiter\\
VM&Versuchsmatrix\\
\end{tabular}